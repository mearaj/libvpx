
\begin{DoxyVerbInclude}
README - 08 March 2021

Welcome to the WebM VP8/VP9 Codec SDK!

COMPILING THE APPLICATIONS/LIBRARIES:
  The build system used is similar to autotools. Building generally consists of
  "configuring" with your desired build options, then using GNU make to build
  the application.

  1. Prerequisites

    * All x86 targets require the Yasm[1] assembler be installed[2].
    * All Windows builds require that Cygwin[3] or MSYS2[4] be installed.
    * Building the documentation requires Doxygen[5]. If you do not
      have this package, the install-docs option will be disabled.
    * Downloading the data for the unit tests requires curl[6] and sha1sum.
      sha1sum is provided via the GNU coreutils, installed by default on
      many *nix platforms, as well as MinGW and Cygwin. If coreutils is not
      available, a compatible version of sha1sum can be built from
      source[7]. These requirements are optional if not running the unit
      tests.

    [1]: http://www.tortall.net/projects/yasm
    [2]: For Visual Studio the base yasm binary (not vsyasm) should be in the
         PATH for Visual Studio. For VS2017 it is sufficient to rename
         yasm-<version>-<arch>.exe to yasm.exe and place it in:
         Program Files (x86)/Microsoft Visual Studio/2017/<level>/Common7/Tools/
    [3]: http://www.cygwin.com
    [4]: http://www.msys2.org/
    [5]: http://www.doxygen.org
    [6]: http://curl.haxx.se
    [7]: http://www.microbrew.org/tools/md5sha1sum/

  2. Out-of-tree builds
  Out of tree builds are a supported method of building the application. For
  an out of tree build, the source tree is kept separate from the object
  files produced during compilation. For instance:

    $ mkdir build
    $ cd build
    $ ../libvpx/configure <options>
    $ make

  3. Configuration options
  The 'configure' script supports a number of options. The --help option can be
  used to get a list of supported options:
    $ ../libvpx/configure --help

  4. Compiler analyzers
  Compilers have added sanitizers which instrument binaries with information
  about address calculation, memory usage, threading, undefined behavior, and
  other common errors. To simplify building libvpx with some of these features
  use tools/set_analyzer_env.sh before running configure. It will set the
  compiler and necessary flags for building as well as environment variables
  read by the analyzer when testing the binaries.
    $ source ../libvpx/tools/set_analyzer_env.sh address

  5. Cross development
  For cross development, the most notable option is the --target option. The
  most up-to-date list of supported targets can be found at the bottom of the
  --help output of the configure script. As of this writing, the list of
  available targets is:

    arm64-android-gcc
    arm64-darwin-gcc
    arm64-darwin20-gcc
    arm64-linux-gcc
    arm64-win64-gcc
    arm64-win64-vs15
    armv7-android-gcc
    armv7-darwin-gcc
    armv7-linux-rvct
    armv7-linux-gcc
    armv7-none-rvct
    armv7-win32-gcc
    armv7-win32-vs14
    armv7-win32-vs15
    armv7s-darwin-gcc
    armv8-linux-gcc
    mips32-linux-gcc
    mips64-linux-gcc
    ppc64le-linux-gcc
    sparc-solaris-gcc
    x86-android-gcc
    x86-darwin8-gcc
    x86-darwin8-icc
    x86-darwin9-gcc
    x86-darwin9-icc
    x86-darwin10-gcc
    x86-darwin11-gcc
    x86-darwin12-gcc
    x86-darwin13-gcc
    x86-darwin14-gcc
    x86-darwin15-gcc
    x86-darwin16-gcc
    x86-darwin17-gcc
    x86-iphonesimulator-gcc
    x86-linux-gcc
    x86-linux-icc
    x86-os2-gcc
    x86-solaris-gcc
    x86-win32-gcc
    x86-win32-vs14
    x86-win32-vs15
    x86-win32-vs16
    x86-win32-vs17
    x86_64-android-gcc
    x86_64-darwin9-gcc
    x86_64-darwin10-gcc
    x86_64-darwin11-gcc
    x86_64-darwin12-gcc
    x86_64-darwin13-gcc
    x86_64-darwin14-gcc
    x86_64-darwin15-gcc
    x86_64-darwin16-gcc
    x86_64-darwin17-gcc
    x86_64-darwin18-gcc
    x86_64-darwin19-gcc
    x86_64-darwin20-gcc
    x86_64-iphonesimulator-gcc
    x86_64-linux-gcc
    x86_64-linux-icc
    x86_64-solaris-gcc
    x86_64-win64-gcc
    x86_64-win64-vs14
    x86_64-win64-vs15
    x86_64-win64-vs16
    x86_64-win64-vs17
    generic-gnu

  The generic-gnu target, in conjunction with the CROSS environment variable,
  can be used to cross compile architectures that aren't explicitly listed, if
  the toolchain is a cross GNU (gcc/binutils) toolchain. Other POSIX toolchains
  will likely work as well. For instance, to build using the mipsel-linux-uclibc
  toolchain, the following command could be used (note, POSIX SH syntax, adapt
  to your shell as necessary):

    $ CROSS=mipsel-linux-uclibc- ../libvpx/configure

  In addition, the executables to be invoked can be overridden by specifying the
  environment variables: CC, AR, LD, AS, STRIP, NM. Additional flags can be
  passed to these executables with CFLAGS, LDFLAGS, and ASFLAGS.

  6. Configuration errors
  If the configuration step fails, the first step is to look in the error log.
  This defaults to config.log. This should give a good indication of what went
  wrong. If not, contact us for support.

VP8/VP9 TEST VECTORS:
  The test vectors can be downloaded and verified using the build system after
  running configure. To specify an alternate directory the
  LIBVPX_TEST_DATA_PATH environment variable can be used.

  $ ./configure --enable-unit-tests
  $ LIBVPX_TEST_DATA_PATH=../libvpx-test-data make testdata

CODE STYLE:
  The coding style used by this project is enforced with clang-format using the
  configuration contained in the .clang-format file in the root of the
  repository.

  Before pushing changes for review you can format your code with:
  # Apply clang-format to modified .c, .h and .cc files
  $ clang-format -i --style=file \
    $(git diff --name-only --diff-filter=ACMR '*.[hc]' '*.cc')

  Check the .clang-format file for the version used to generate it if there is
  any difference between your local formatting and the review system.

  See also: http://clang.llvm.org/docs/ClangFormat.html

SUPPORT
  This library is an open source project supported by its community. Please
  email webm-discuss@webmproject.org for help.

\end{DoxyVerbInclude}
 